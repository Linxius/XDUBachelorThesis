%% 正文是作者对研究工作的详细表述。它占全文的绝大部分,其内容应包括:问题的
%% 提出,研究工作的基本前提、假设和条件;模型的建立,实验方案的拟定;基本要领和
%% 理论基础;设计计算的主要方法和内容;实验方法、内容及其分析;理论论证,理论在
%% 课题中的应用,课题得出的结果,以及对结果的讨论等。根据毕业设计(论文)题目的
%% 性质,一般情况下,正文可能仅包含上述的一部分内容。撰写正文部分的具体要求如下
%% 所述。
%% ( 1)理论分析部分应写明所作的假设及其合理性,所用的分析方法、计算方法、
%% 实验方法等哪些是别人用过的,哪些是自己改进的,哪些是自己创造的,以便指导教师
%% 审查和纠正,篇幅不宜过多,应以简练的文字概略地表述。
%% ( 2)对于用实验方法研究的课题,应具体说明实验用的装置、仪器的性能, 并应
%% 对所用装置、仪器做出检验和标定。对实验的过程和操作方法,力求叙述简明扼要,对
%% 人所共知的内容或细节内容不必详述。
%% 对于经理论推导达到研究目的的课题,内容要精心组织,做到概念准确,判断推理
%% 符合客观事物的发展规律,符合人们对客观事物的认识习惯,换言之,要做到言之有序,
%% 言之有理,以论点为中心,组成完整而严谨的内容整体。
%% ( 3)结果与讨论是全文的心脏,一般要占较多篇幅,在撰写时对必要而充分的数
%% 据、现象、认识等要作为分析的依据写进去。在对结果作定性和定量分析时,应说明数
%% 据的处理方法以及误差分析,说明现象出现的条件及其可证性,交代理论推导中认识的
%% 由来和发展,以便他人以此为依据进行核实验证。对结果进行分析后得出的结论,也应
%% 说明其适用的条件与范围。此外,适当运用图、表作为结果与分析,是科技论文通用的
%% 一种表达方式。

%\input{chapter1}